\section{Implementáció}

\subsubsection{Fejlesztő környezet}

Mind a kliens, mind a szerver oldal implementálása Java nyelven történik. A kliens oldal az Ericsson Service Development Studio (SDS) 4.1 fej\-lesz\-tő\-esz\-köz segítségével valósult meg. Az SDS egy szabadon hozzáférhető fejlesztő környezet, amely képes emulálni az IMS hálózat főbb funkcionális egységeit, így lehetőséget teremt IMS hálózatra szánt alkalmazások fejlesztésére, valamint azok tesztelésére. Az SDS a nyílt forráskódú Eclipse fejlesztői környezetre épül, így az egész eszköz bármilyen, Microsoft Windows XP operációs rendszert futtató PC-n használható. Szabványosított API-kat használhatunk kliens és szer\-ver\-ol\-da\-li programozáshoz, valamint a kommunikációs folyamatok, szolgáltatások megvalósításához egyaránt. A fejlesztői keretrendszerbe beépített tesztelési egységek lehetőséget nyújtanak végpontól-végpontig tartó tesztelésre. Az SDS tartalmaz előre elkészített funkciókat is, amelyeket a fejlesztők 
több-kevesebb sikerrel %???
 felhasználhatnak az új szolgálatások elkészítése során. Ilyen funkció például a jelenlét és csoportkezelés (Presence and Group Management), az IMS üzenetküldés (IMS-Messaging), valamint az IP hálózat feletti hangátvitel (Voice Over IP).
