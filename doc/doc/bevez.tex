
\section{Bevezetés}

Ide jön a bevezetés... Íme egy hivatkozás~\cite{KonyvId}. Még egy~\cite{sds_tech_desc}

\subsection{Motiváció}

Napjaink infokommunikációs hálózatainak fejlődési tendenciája azt mutatja, hogy az egymástól elszigetelt,
különálló szolgáltató hálózatok helyett egy konvergált, egységes hálózat van kialakulóban. Egységes hálózaton azt értem, hogy a felhasználók azonos módon érhetik el a szolgáltatásokat függetlenül attól, hogy éppen milyen hálózatról, milyen eszközzel -- mobil vagy fix -- kívánják ezt megtenni. Ehhez egyrészt az szükséges, hogy a fix és a mobil hálózatok konvergenciája megvalósuljon (fixed-mobile convergence), másrészt a szolgáltatások azonos hálózatra történő konvergeciájára is szükség van (multiple play). Utóbbi azt jelenti, hogy egy szolgáltató ugyanazon a hálózaton képes különböző szolgáltatásokat nyújtani az előfizetőknek. Itt jut szerephez az IMS (IP Multimedia Subsystem). Az IMS-ről részletesebben a \ref{sec:ims}.~fejezetben lesz szó.

Korunkban egyre többen használnak különféle elektronikus kommunikációs formákat, különféle környezetben, különféle eszközökkel. Ezek az eszközök jellemzően más-más módon kommunikálnak egymással, amely sok esetben jelentős kompatibilitási problémákhoz vezet.

\subsection{Célkitűzés}

A feladat célkitűzése egy olyan rendszer megalkotása az IMS rendszerben, melynek
segítségével a felhasználónak lehetősége nyílik multimédia üzenet küldésére felhasználók egy előre megadott csoportjának. Fontos szempont az is, hogy ha egy felhasználó az üzenetküldés idejében éppen nem kapcsolódik a hálózathoz, később akkor is megkapja a neki szánt üzeneteket.

\subsection{Az IP Multimédia Alrendszer}
\label{sec:ims}

Az IMS egy jelentős lépés a ``all-IP'' hálózati architektúra felé. Felelős a hagyományos telefon szolgáltatásokért illetve az új, multimédia szolgáltatásokért is. Egységes jelzési hálózatot nyújt az eszközöknek, amely jel\-zés\-há\-ló\-zat különválik a tartalom átvitelért felelős hálózattól. Ennek következtében a jelzéshálózat független lesz magától a tartalomtól. Az IMS-ben a jelzési protokollnak a  SIP (Session Initiaton Protocol) protokollt választották. A SIP egy alkalmazás rétegbeli protokoll, amely lehetővé teszi különböző típusú multimédia kapcsolatok felépítését, módosítását, befejezését az IP (Internet Protocol) hálózatok felett.


