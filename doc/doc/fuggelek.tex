
\appendix

\section{Függelék}
\label{sec:sipMsgXmlFuggelek}

\noindent
{\bf SIP MESSAGE üzenet törzsében átvitt XML tartalom sémája}
\fontsize{10}{10}
\begin{verbatim}
<?xml version="1.0" encoding="UTF-8"?>
<xs:schema xmlns:xs="http://www.w3.org/2001/XMLSchema"> 
<xs:element name="information">
   <xs:complexType>
      <xs:sequence>
          <!-- A üzenet típusa -->
          <xs:element name="request_type">
             <xs:simpleType>
                <xs:restriction base="xs:string">                        
                   <xs:enumeration value="STATUS_UPDATE"/>
                   <xs:enumeration value="NOTIFY"/>
                   <xs:enumeration value="DELETE_MESSAGE"/>
                   <xs:enumeration value="MESSAGE_DATA"/>
                </xs:restriction>
             </xs:simpleType>
          </xs:element>                                    
          <xs:element name="message" maxOccurs="unbounded">
             <xs:complexType>                    
                <xs:sequence>
                   <!-- STATUS_UPDATE esetén regisztráció vagy törlés -->
                   <xs:element name="status" type="xs:string" minOccurs="0"/>
                   <!-- A küldö adatai -->
                   <xs:element name="sender" minOccurs="0">
                      <xs:complexType>
                         <xs:sequence>
                            <xs:element name="name" type="xs:string"/>
                            <xs:element name="sip_uri" type="xs:string"/>
                         </xs:sequence>
                      </xs:complexType>
                   </xs:element>
                   <!-- Az üzenet azonosítója -->
                   <xs:element name="msrp_id" type="xs:string" minOccurs="0"/>
                   <!-- A multimédia tartalom típusa -->
                   <xs:element name="mime_type" type="xs:string" minOccurs="0"/>
                   <!-- Az üzenet tárgya -->
                   <xs:element name="subject" type="xs:string" minOccurs="0"/>
                   <!-- A küldés dátuma -->
                   <xs:element name="sent_at" type="xs:date" minOccurs="0"/>
                   <!-- Címzettek listája -->
                   <xs:element name="recipients" minOccurs="0">
                      <xs:complexType>
                         <xs:sequence>
                            <xs:element name="recipient" maxOccurs="unbounded">
                               <xs:complexType>
                                  <xs:sequence>
                                     <xs:element name="name" type="xs:string"/>
                                     <xs:element name="sip_uri" type="xs:string"/>
                                  </xs:sequence>
                               </xs:complexType>
                            </xs:element>
                         </xs:sequence>                    
                      </xs:complexType>                
                   </xs:element>            
                </xs:sequence>
             </xs:complexType>
          </xs:element>            
       </xs:sequence>
    </xs:complexType>
</xs:element>
</xs:schema>
\end{verbatim}
\fontsize{12}{12} 


\section{Függelék}
\label{sec:sql_utasitasok_fuggelek}

\noindent
{\bf Az adatbázis táblákat létrehozó SQL utasítások}

\fontsize{10}{10}
\begin{verbatim}
CREATE TABLE MESSAGES
(
  ID INT NOT NULL AUTO_INCREMENT,
  MSRP_MESSAGE_ID VARCHAR(200) NOT NULL,
  CONTENT BLOB
  SENDER_NAME VARCHAR(150)
  SENDER_SIP_URI VARCHAR(150)
  SENT_AT DATE
  SUBJECT VARCHAR(250)
  CONTENT_TYPE VARCHAR(50),
  CONSTRAINT pk_messages PRIMARY KEY(ID),
  CONSTRAINT uq_msrp_message_id UNIQUE(MSRP_MESSAGE_ID)
) ENGINE=InnoDB DEFAULT CHARSET=utf8;
\end{verbatim}
\fontsize{12}{12}

\fontsize{10}{10}
\begin{verbatim}
CREATE TABLE RECIPIENTS
(
  ID INT NOT NULL AUTO_INCREMENT,
  MESSAGE_ID INT NOT NULL,
  NAME VARCHAR(150)
  SIP_URI VARCHAR(150)
  DELIVERY_STATUS VARCHAR(30)
  CONSTRAINT pk_recipients PRIMARY KEY (ID)
  CONSTRAINT fk_messages FOREIGN KEY (MESSAGE_ID) REFERENCES MESSAGES(ID)
) ENGINE=InnoDB DEFAULT CHARSET=utf8;
\end{verbatim}
\fontsize{12}{12}


