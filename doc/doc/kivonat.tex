
\section*{Kivonat}
\addcontentsline{toc}{section}{Kivonat}

Napjainkban az Internet gyors terjedése következtében a kommunikációs formák kiszélesedése, egyre bővülő választéka figyelhető meg. A már meglévő, illetve kialakulóban lévő, különböző kommunikációs formákat célszerű egységesen kezelni. Az IMS erre az egységes kezelésre kínál megoldást, aminek alapjául az IP protokoll szolgál. Vezérlési struktúrája révén széleskörű, rugalmas lehetőséget teremt új szolgáltatások gyors bevezetésére.

A szóban forgó kommunikációs formák közé tartozik a csoportos üzenetküldés is. A felhasználók gyorsan, egyszerűen, lehetőség szerint integrált megvalósításokkal szeretnének információkat megosztani az ismerőseikkel, aminek során valamilyen értéknövelt szolgáltatást használhatnak. Ilyen hozzáadott funkció lehet az egyszerű szöveges üzenet helyett tetszőleges multimédia tartalom küldésének lehetősége.

Munkám során megvizsgáltam, hogy a csoportos multimédia üzenetküldést megvalósító szolgáltatásnak milyen követelményeknek kell megfelelnie, a szóbajövő megoldási lehetőségeket részletesen tárgyaltam, kitérve azok előnyeire, valamint hátrányaira. A célnak leginkább megfelelő megvalósítási módszer kiválasztása, annak részletes megtervezése után az üzenetküldő szolgáltatást, továbbá a szolgáltatást használó kliens alkalmazást Java nyelven implementáltam. A munka eredményeként létrejött szolgáltatást tesztelem, hogy az milyen mértékben felel meg a specifikációnak. A tesztelési folyamat ismertetése után olyan továbbfejlesztési lehetőségeket, új funkciókat mutatok be, amik a szolgáltatás használhatóságát javíthatják.
