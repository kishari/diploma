
\section{Összefoglalás}
\label{sec:osszefoglalas}

Az infokommunikáció világában megfigyelhető változások, trendek, a hálózatok, eszközök konvergenciájának, valamint a kommunikációs formák térhódításának következtében megnövekedett az igény új, a kommunikáció során használható integrált szolgáltatások létrehozására. Jelen diplomaterv feladata egy olyan csoportos multimédia üzenetküldő szolgáltatás tervezése, valamint implementálása volt, aminek segítségével a felhasználók hang, kép illetve videó üzenetet küldhetnek ismerőseik egy meghatározott csoportjának. A feladat részét képezte a szolgáltatást használó kliens oldali alkalmazás megvalósítása is.

A megvalósítási folyamat során ismertettem a szolgáltatáshoz szükséges technológiai ismereteket, bemutattam, valamint összehasonlítottam az egyes megoldási lehetőségeket. A célnak leginkább megfelelő módszert követve megterveztem a szolgáltatás felépítését, a kommunikációs folyamatok működését, az üzenetküldés során szükséges adatok átvitelének módját, illetve formáját. 

A tervezési fázis befejezése után Java nyelven implementáltam egyrészt magát a szolgáltatást, továbbá a szolgáltatás funkcióit használó kliens alkalmazást. A tartalomátvitel során használt MSRP protokoll az üzenetküldéshez szükséges alapvető részeit megvalósítottam. A fejlesztés eredményeként létrejött szerver oldali szolgáltatást, valamint a kliens alkalmazás fontosabb részeinek felépítését osztálydiagrammok formájában szemléltettem. A szolgáltatás egy SIP alkalmazás szerveren, míg a kliens oldal egy megfelelően beállított ICP-t tartalmazó Microsoft Windows operációs rendszeren futott.

A fejlesztés befejeztével a létrejött szolgáltatásban az elvárt funkciókat az SDS tesztelést támogató eszközeivel ellenőriztem. A célkitűzésnek megfelelően a szolgáltatás sikeresen megvalósítja a csoportos multimédia üzenetküldés elvárt működését. Végezetül továbbfejlesztési lehetőségeire tettem ajánlást, amikkel véleményem szerint a szolgáltatás a felhasználók számára használhatóbbá, gazdagabbá tehető.