

\section{Tervezés}

\subsection{Bevezető}

Jelen rendszerterv a leírt rendszer részletes megvalósítását
tartalmazza, a tervezés egyes fázisait külön tárgyalja, a felmerülő
problémákra minél pontosabb és részletesebb megoldásokat keresve.

\subsection{A megvalósítási lehetőségek}

\subsection{Funkcionális terv}


A következő fejezetekben az egyes elemek funkciójának részletes tárgyalása
következik.

\subsubsection{A kliens pc}
\label{sec:kliens_pc}


\subsubsection{Alkalmazás szerver}


\subsubsection{Adatbázis szerver}
\label{sec:adatbszerver}

A küldő felhasználótól az alkalmazás szerveren keresztül érkező
multimédia üzenetek tárolását valósítja meg.

Tárolásra kerülnek a felhasználótól érkező üzenet adatai. Ezek közé tartozik
a feladó azonosítója (SIP URI), a címzettek azonosítói (SIP URI), valamint maga a multimédia üzenet.

\subsubsection{Fejlesztő környezet}

Mind a kliens, mind szerver oldal implementálása Java nyelven történik. A kliens oldal megvalósítása az Ericsson cég által nyújtott Service Development Studio 4.0 fejlesztő eszköz segítségével történik. Az eszköz képest az IMS rendszer fő funkcionális elemeit emulálni.


\subsection{Tesztelési terv}
\label{sec:tesztelesi_terv}

A rendszer tervezésének fontos részét képezi a funkcionális tesztelés. Leegyszerűsíti a megvalósítást ill. segít elkerülni olyan hibákat, melyekre az imp\-le\-men\-tá\-lás során talán nem derülne fény. A rendszer étékelésének alapjául szolgál.

Teszteket érdemes modulonként végezni, ami után - feltételezve, hogy az egyes mo\-du\-lok működése megfelelő - a teljes rendszer tesztelése következhet. %Ehhez meg kell határozni, hogy melyik bemeneti adatra mi az elvárt viselkedés. Unit tesztek.
Az egyes modulok és feladatainak részletes leírása \aref{sec:interfesz_terv}.~fejezetben található. Az alábbiakban ezen modulok tesztelésének rövid leírása, majd a globális tesztelés következik.

\subsubsection{Modulok tesztelése}

A kliens és szerver közötti hálózaton történő kommunikációt megvalósító üzenetek előállításának ill. a beérkező üzenetek elemzésének tesztelése szintén elengedhetetlen a helyes működéshez.

\subsubsection{Globális tesztelés}

\subsection{Üzemeltetési terv}
\label{sec:uzemeltetesi_terv}

\subsection{Interfész terv}
\label{sec:interfesz_terv}

Két fő komponensre különül el a rendszer fejlesztése: kliens-
ill. szerveroldali részre.

\subsubsection{Kliens}
\label{sec:kliensinterfesz}

\subsubsection{Szerver}
\label{sec:szerverinterfesz}

\subsection{Adatbázis terv}

A következő fejezetek a kommunikációs üzenetek szerkezetét valamint a használt adatbázis felépítését tárgyalják.


\subsubsection{Kommunikációs üzenetek}
\label{sec:kommuzenetek}

\subsubsection{Az adatbázis}
\label{sec:adatb}

\subsection{Megvalósítási terv}
\label{sec:megvalositas}

\subsubsection{A kliens megvalósítása}

\subsubsection{A szerver megvalósítása}
\label{sec:szervermegvalositas}


\subsection{Összefoglalás}

