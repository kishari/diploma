

\section{Tervezés}

\subsection{Bevezető}

Jelen rendszerterv a leírt rendszer részletes megvalósítását
tartalmazza, a tervezés egyes fázisait külön tárgyalja, a felmerülő
problémákra minél pontosabb és részletesebb megoldásokat keresve.

\subsection{A megvalósítási lehetőségek}

Ebben a részben megvizsgálom, hogy az IMS használata esetén milyen lehetőségek vannak a csoportos üzenetküldés megvalósítására. A megoldások tárgyalása során kitérek azok előnyeire, a hátrányaira, illetve a felmerülő problémákra.

\subsubsection{SIP MESSAGE üzenetek használata}

Első megoldásként kézenfekvőnek tűnhet, hogy a SIP protokoll által nyújtott MESSAGE típusú üzenet törzsében küldjük el a multimédia üzenetet a címzetteknek. Ennek a megoldásnak az az előnye, hogy nem kell másik protokollt használni az üzenetek átviteléért, hanem a SIP protokoll nyújtotta funkciókat alkalmazzuk. A megoldás előnye viszont eltörpül a hátrányok mellett. Először is a MESSAGE típusú üzenetnek nem adható meg egynél több címzett, így minden címzettnek különálló MESSAGE üzenetben kellene elküldeni ugyanazt a tartalmat. Ez a megszorítás redundanciához vezet, mivel a feladónak ugyanazt az üzenetet minden címzettnek külön el kell küldenie, ami a válaszidőt is jelentősen megnövelné. Erre a problémára megoldást jelentene, ha az üzenetet egy alkalmazás szerveren keresztül küldenénk, és a szerver továbbítaná azt a címzetteknek. Így a küldő csak egyszer küldené el az üzenetet. Ahhoz, hogy ez a megvalósítás működhessen, a felhasználókból csoportokat kellene létrehozni, és a MESSAGE üzenetet a címzettek URI-ja (Uniform Resource Identifier) helyett a csoport URI-jával címeznénk. Amikor az alkalmazás szerver megkapná ez az üzenetet, a csoport azonosító alapján megkeresné azokat a felhasználókat, akik tagjai a csoportnak, és mindegyik csoporttagnak egyesével elküldené az üzenet másolatát. 
További problémát jelent, ha küldött multimédia mérete meghaladja a MESSAGE üzenet törzsébe maximálisan megadható 1300 oktetet\footnote{RFC 3428 - Session Initiation Protocol Extension for Instant Messaging}. Ilyenkor a tartalmat a küldő oldalon kisebb darabokra kellene tördelni, a darabokat külön üzenetekben elküldeni, majd a vevő oldalon a kisebb részekből a teljes üzenetet rekonstruálni. A probléma gyökere, hogy a MESSAGE üzenettípus nem támogatja, hogy a vevő képes legyen az eredeti üzenetet előállítani az üzenet részletekből. A MESSAGE üzenet fejlécében nincs lehetőség olyan paramétert megadni, amiből a vevő el tudja dönteni, hogy a beérkező MESSAGE üzenetek közül melyik hordozza ugyanazon tartalom egy-egy darabját, és melyik nem. További gond, hogy az üzenetek nem sorszámozottak, így azt eredeti küldési sorrendet sem tudnánk visszaállítani. Ezek a problémák abból fakadnak, hogy a MESSAGE üzenet használatát elsősorban rövid szöveges üzenetek továbbítására találták ki, és nem nagy multimédia tartalmakhoz. 
A megoldás hátrányainak sorát bővíti az is, hogy nem támogatott a késleltetett üzenetküldést sem. Utóbbi problémát -- hasonlóan az elsőhöz -- orvosolhatnánk az alkalmazás szerver használatával, amely eltárolná azokat az üzeneteket, amelyek a nem elérhető címzetteknek mennek, és akkor kézbesítené azt, amikor a címzett elérhetővé válik.(IDE MÉG RFC3428-ból PÁR DOLOG)
(Leírni még, hogy a vevők válasza többször jönne, stb...)

\subsubsection{Több címzettel rendelkező üzenetek használata}
Az az eset, amikor az üzenet fejlécében benne van minden címzett. Jelenleg CSCF nem támogatja, forkolni lehet az üzenetet attól függően, hogy merre kell továbbítani azt, stb... Leírni rendesen!

\subsubsection{Üzenet továbbítása RTP protokoll segítségével}
Ide kell MRF, AS stb... Leírni!

\subsubsection{Üzenet továbbítása MSRP protokollal}
Ezt csinálom én.







\subsection{Funkcionális terv}


A következő fejezetekben az egyes elemek funkciójának részletes tárgyalása
következik.

\subsubsection{A kliens pc}
\label{sec:kliens_pc}


\subsubsection{Alkalmazás szerver}


\subsubsection{Adatbázis szerver}
\label{sec:adatbszerver}

A küldő felhasználótól az alkalmazás szerveren keresztül érkező
multimédia üzenetek tárolását valósítja meg.

Tárolásra kerülnek a felhasználótól érkező üzenet adatai. Ezek közé tartozik
a feladó azonosítója (SIP URI), a címzettek azonosítói (SIP URI), valamint maga a multimédia üzenet.

\subsubsection{Fejlesztő környezet}

Mind a kliens, mind szerver oldal implementálása Java nyelven történik. A kliens oldal megvalósítása az Ericsson cég által nyújtott Service Development Studio 4.1 fej\-lesz\-tő eszköz segítségével történik. Az eszköz képest az IMS rendszer fő funkcionális elemeit emulálni.


\subsection{Tesztelési terv}
\label{sec:tesztelesi_terv}

A rendszer tervezésének fontos részét képezi a funkcionális tesztelés. Leegyszerűsíti a megvalósítást ill. segít elkerülni olyan hibákat, melyekre az imp\-le\-men\-tá\-lás során talán nem derülne fény. A rendszer étékelésének alapjául szolgál.

Teszteket érdemes modulonként végezni, ami után - feltételezve, hogy az egyes mo\-du\-lok működése megfelelő - a teljes rendszer tesztelése következhet. %Ehhez meg kell határozni, hogy melyik bemeneti adatra mi az elvárt viselkedés. Unit tesztek.
Az egyes modulok és feladatainak részletes leírása \aref{sec:interfesz_terv}.~fejezetben található. Az alábbiakban ezen modulok tesztelésének rövid leírása, majd a globális tesztelés következik.

\subsubsection{Modulok tesztelése}

A kliens és szerver közötti hálózaton történő kommunikációt megvalósító üzenetek előállításának ill. a beérkező üzenetek elemzésének tesztelése szintén elengedhetetlen a helyes működéshez.

\subsubsection{Globális tesztelés}

\subsection{Üzemeltetési terv}
\label{sec:uzemeltetesi_terv}

\subsection{Interfész terv}
\label{sec:interfesz_terv}

Két fő komponensre különül el a rendszer fejlesztése: kliens-
ill. szerveroldali részre.

\subsubsection{Kliens}
\label{sec:kliensinterfesz}

\subsubsection{Szerver}
\label{sec:szerverinterfesz}

\subsection{Adatbázis terv}

A következő fejezetek a kommunikációs üzenetek szerkezetét valamint a használt adatbázis felépítését tárgyalják.


\subsubsection{Kommunikációs üzenetek}
\label{sec:kommuzenetek}

\subsubsection{Az adatbázis}
\label{sec:adatb}

\subsection{Megvalósítási terv}
\label{sec:megvalositas}

\subsubsection{A kliens megvalósítása}

\subsubsection{A szerver megvalósítása}
\label{sec:szervermegvalositas}


\subsection{Összefoglalás}

